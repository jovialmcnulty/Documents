More than ever the need for engineers has increased. Society is moving towards automation and with it comes a need for engineers. To fill this need, it is the responsibility of engineers to attract the next generation of engineers to the field. One of the best way of attracting people to the profession is by building projects that people can see and experience. This is the basis for the Jenga playing robot.

The societal need for a Jenga robot is non-existent. We aren't designing this with the expectation of selling it for profit or creating patents (although, we aren't against these outcomes). Our intention for developing this robot is to create a robot that will entertain and interact with the public by playing the family-friendly game of Jenga.

Plenty of projects have been designed with the intention of swaying the opinions of the public towards engineering. There has been Rubik's cube solving robots, chess playing robots, and even other Jenga robots that have come before us. The goal of our project is no different from theirs, we plan on showcasing our robot to gain public interest for engineering rather than developing it for monetary gain. 

Our sponsor, Cloud9 Perception, is looking to build excitement for the engineering field as well as have a centerpiece to draw people into their conference booths. We will use their vision solutions and resources to help develop the robot to be able interact with conference guests. This interaction should be interesting enough to draw them into their booth and help bring them business as well as build awe towards engineering.

The Jenga Playing Robot has been done plenty of times before, but unlike our predecessors we intend to have the robot actually play with a player. The design of previous Jenga playing robots have the awe of a robot selecting from a tower of blocks correctly, but none before us have played Jenga against human opponents. We will surpass our predecessors by building a Jenga playing robot that is more robust and capable of competition.
