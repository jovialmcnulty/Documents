The stakeholders of the project will be our team, consisting of five members, who will be developing and seeing the project to completion. Cloud9 Perception, our sponsor, also has a stake in this project as they would like a product that showcases their 3D scanning technology and will be donating 3D scanning parts to this Jenga playing robot project. At a higher level, The University of Texas at Arlington has a stake in this project, as the success of the Jenga playing robot determines the preparedness of the students of the Department of Computer Science Engineering.

The point of contact for both our sponsor and our customer will be Dr. McMurrough, as the course professor and owner of Cloud9 Perception. We will be in contact with Dr. McMurrough in order to determine if we are going on the right path in developing the Jenga playing robot.

The team members consist of Gabe Comer, Carlos Crane, Joe Cloud, Sammy Hamwi, and Maxwell Sanders. Joe Cloud, being a computer engineer as well as experienced with robotics will be the lead on most of the hardware to software interface challenges that we will experience. Carlos Crane and Sammy Hamwi, having proficient front-end and user experience will be able to take charge of most of the user interface and design aspects of our project. Gabe Comer and Maxwell Sanders, along with Carlos and Sammy, will take care of 3D sensing, modeling, and algorithmic development of the gameplay and robotic arm interaction. Maxwell Sanders has had experience coding for the UR5 arm specifically, and will be the lead in developing that component.

Our team is going to rotate the product owner and the scrum master periodically. This will give all of us an opportunity at leadership as well as let us see the project from different roles and perspectives. 