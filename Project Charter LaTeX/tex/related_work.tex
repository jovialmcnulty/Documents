Our team found several implementations of Jenga playing robot arms that already exist, all of which were created for the purpose of academic research. However, these implementations would not be ideal for trade show demonstrations for various reasons, including lack of full automation, lack of interactivity, and excessive downtime between each round of play.

For instance, the Jenga playing manipulator created by Kroger et al. \cite{kroger2008manipulator} does not play against a human opponent. While their implementation was extremely successful in terms of the number of block pulls (the record highest tower was over 28 levels), for safety purposes, their implementation played against itself. This implementations goal was to stack the highest tower with no opponent. We would like for our implementation to play against human opponents in a controlled and safe manor.

Another implementation, created by Chia-Hung Lin \cite{lin2018robot} uses the same type of UR5 robotic arm as our project. However, Lin's project takes up to 60 seconds to scan the tower and make it's next move. Since our project is meant to be an exciting demonstration used to attract people at trade shows, we want to quicken the pace. Furthermore, Lin uses AR tags to mark important locations, like the edges of the tower. Since part of our goal is to demonstrate the capabilities of Cloud 9 Perception's vision hardware, we would like to avoid using AR tags, instead using computer vision to orient the robot around the tower.

The implementations by Kimura et al. \cite{kimura2010force} and Wang et al. \cite{wang2009robot} both use detailed physics simulations to decide which block to pull next. Their research could be quite useful when creating the algorithm that decides which block to pull next. However, in the case of Kimura et al., a human must assist the robot in pulling blocks by spinning a rotating platform that the Jenga tower rests on. While their implementation can nearly match average human players in number of block-pulls, this level of human assistance is not acceptable for our project. Our goal is to enable the UR5 Jenga playing robot to play a full game of Jenga against an opponent without assistance. In the case of Wang et al., they were able to detect movements in the tower that could lead to collapse, allowing them to rule out the more dangerous block pulls (in contrast to Kroger et al., who pulled blocks at random \cite{kroger2008manipulator}). However, their  This robot did not stack removed blocks on the top of the Jenga tower, and was therefore not a suitable opponent for a game of Jenga. 
