%Include a header paragraph specific to your product here. Customer requirements are those required features and functions specified for and by the intended audience for this product. This section establishes, clearly and concisely, the "look and feel" of the product, what each potential end-user should expect the product do and/or not do. Each requirement specified in this section is associated with a specific customer need that will be satisfied. In general Customer Requirements are the directly observable features and functions of the product that will be encountered by its users. Requirements specified in this section are created with, and must not be changed without, specific agreement of the intended customer/user/sponsor.
This section covers the features and functions that can be expected from the UR5 Jenga-Playing Robot. Every requirement in this section is associated with a specific customer need that we need to satisfy. Our customers include: ourselves, our sponsor, our class, and the college of engineering. The features covered in this section are directly observable and should be transparent to the customer who expects the requirement. Any of the requirements below should not be changed without the agreement of the affected customer.

\subsection{Input Interface}
\subsubsection{Description}
The user interface of the UR5 will be simple and intuitive. It shall be a button that determines when the player move is completed, and the robot move is ready to begin. 
\subsubsection{Source}
Maxwell
\subsubsection{Constraints}
The button should not take up a considerable amount of space, and should not interfere with the game. Pressing the button should not shake the tower.
\subsubsection{Standards}
N/A
\subsubsection{Priority}
High

\subsection{Output Interface}
\subsubsection{Description}
The output interface will be a LED that indicates whether or not the robot has completed its turn.
\subsubsection{Source}
Maxwell
\subsubsection{Constraints}
The LED will not be blinding, and should not interfere with the vision system of the robot as well as the vision of the player. The LED should not affect the gameplay.
\subsubsection{Standards}
N/A
\subsubsection{Priority}
Low

%Added by Gabe
\subsection{Custom Jenga Tower}
\subsubsection{Description}
The UR5 Jenga-Playing Robot shall include with it for use in demonstrations, one custom Jenga tower. The tower shall be comprised of 54 3D printed blocks, each block. 
\subsubsection{Source}
Gabriel Comer
\subsubsection{Constraints}
3D printing takes a lot of time, so having to print a new tower if our first attempt is too small, too heavy, or has too much static friction, could add significant delays. Also, printing 54-block towers could become costly as we try different block types and iterate through block size and material options.
\subsubsection{Standards}
N/A
\subsubsection{Priority}
Medium

\subsection{Turn Autonomy}
\subsubsection{Description}
The UR5 Jenga-Playing Robot shall pull blocks from the tower and place them on the top without intervention by the user. Once the users has signalled the end of their turn through the button, the user will not be required to touch the tower, or interact with the display in any way, until their next turn. User intervention might improve the UR5 Jenga-Playing Robot's performance, but it would also decrease the excitement of the game being played.
\subsubsection{Source}
Gabriel Comer%it says we can use 'specified team member by name', so I just put my name
\subsubsection{Constraints}
Lack of user intervention will likely decrease the block pulling success rate of the UR5 Jenga-Playing Robot. In cases where the tower is crooked, or leaning to one side, the UR5 Jenga-Playing Robot may have difficulty obtaining accurate scans of the tower, and pulling blocks without knocking the tower down.
\subsubsection{Standards}
N/A
\subsubsection{Priority}
High


%\subsubsection{Description}
%Detailed requirement description...
%\subsubsection{Source}
%Source
%\subsubsection{Constraints}
%Detailed description of applicable constraints...
%\subsubsection{Standards}
%List of applicable standards
%\subsubsection{Priority}
%Priority

%\subsection{Requirement Name}
%\subsubsection{Description}
%A detailed description of the feature/function that satisfies the requirement. For example: \textit{The GUI background will be slate blue. This specific color is required in order to ensure that the GUI matches other similar software products offered by the customer. Slate blue is specified as \#007FFF, using six-digit hexadecimal color specification.} It is acceptable and advisable to include drawings/graphics in the description if it aids understanding of the requirement.
%\subsubsection{Source}
%The source of the requirement (e.g. customer, sponsor, specified team member (by name), federal regulation, local laws, CSE Senior Design project specifications, etc.)
%\subsubsection{Constraints}
%A detailed description of realistic constraints relevant to this requirement. Economic, environmental, social, political, ethical, health \& safety, manufacturability, and sustainability should be discussed as appropriate.
%\subsubsection{Standards}
%A detailed description of any specific standards that apply to this requirement (e.g. \textit{NSTM standard xx.xxx.x. color specifications \cite{Rubin2012}}. Standards exist for practically everything (ATC standard fuses, IEEE 802.15.4 embedded wireless, TLS 1.3 encryption, etc.), so be sure that you research and document which ones will be followed in meeting this requirement.
%\subsubsection{Priority}
%The priority of this requirement relative to other specified requirements. Use the following priorities:
%\begin{itemize}
%\item Critical (must have or product is a failure)
%\item High (very important to customer acceptance, desirability)
%\item Moderate (should have for proper product functionality);
%\item Low (nice to have, will include if time/resource permits)
%\item Future (not feasible in this version of the product, but should be considered for a future release).
%\end{itemize}

%\subsection{Requirement Name}
%\subsubsection{Description}
%Detailed requirement description...
%\subsubsection{Source}
%Source
%\subsubsection{Constraints}
%Detailed description of applicable constraints...
%\subsubsection{Standards}
%List of applicable standards
%\subsubsection{Priority}
%Priority

